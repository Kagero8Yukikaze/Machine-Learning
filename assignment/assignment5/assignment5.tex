\documentclass[12pt, a4paper]{article}

\usepackage{ctex} % 使用ctex宏包支持中文
\usepackage{geometry} % 页面设置
\usepackage{amsmath, amssymb, ntheorem} % 数学公式
\usepackage{graphicx} % 插入图片
\usepackage{enumitem} % 列表环境
\usepackage{hyperref} % 超链接
\usepackage{tikz}
\usepackage{tikz-qtree}

% 页面设置
\geometry{left=2.5cm, right=2.5cm, top=2.5cm, bottom=2.5cm}

% 标题信息
\title{课后练习5}
\author{}
\date{}

\begin{document}

\maketitle % 显示标题

\section{问题一}

\subsection{}

我们已知
\begin{align*}
    X&=\begin{pmatrix}
        3&4\\5&6\\1&2\\4&3\\2&5
    \end{pmatrix}\quad
    \bar{x}=\begin{pmatrix}
        3&4
    \end{pmatrix}
\end{align*}
因此
\begin{align*}
    \hat{X}&=\begin{pmatrix}
        0&0\\2&2\\-2&-2\\1&-1\\-1&1
    \end{pmatrix}\\
    \Sigma&=\frac{1}{n}\hat{X}^T\hat{X}=\begin{pmatrix}
        2&1.2\\1.2&2
    \end{pmatrix}
\end{align*}
对$\Sigma$进行特征值分解得到
\begin{align*}
    \Sigma = W^T \Lambda W \approx 
    \begin{pmatrix}
        -0.707&0.707\\0.707&0.707
    \end{pmatrix}
    \begin{pmatrix}
        0.8&0\\0&3.2
    \end{pmatrix}
    \begin{pmatrix}
        -0.707&0.707\\0.707&0.707
    \end{pmatrix}
\end{align*}
投影矩阵$W$及投影后的数据为
\begin{align*}
    W&=\begin{pmatrix}
        -0.707&0.707\\0.707&0.707
    \end{pmatrix}\\
    X'&=XW=\begin{pmatrix}
        0.707&4.949\\0.707&7.777\\0.707&2.121\\-0.707&4.949\\-2.121&4.949
    \end{pmatrix}
\end{align*}

\begin{figure}
    \centering
    \includegraphics*{img/a5_1.jpg}
    \caption{1.1}
\end{figure}

\subsection{}

保证不同维度正交,可以消除不同维度之间的相互影响,使得不同维度之间相互独立,互不影响


\section{问题二}

\subsection{}

1. $\mu_1=(1,1)\quad\mu_2=(6,7)$

2. $(1,1)(1,2)(2,1)(3,4)$被分配到$\mu_1$,$(6,7)(7,6)$被分配到$\mu_2$

3. $\mu_1=(1.75,2)\quad\mu_2=(6.5,6.5)$

4. $(1,1)(1,2)(2,1)(3,4)$被分配到$\mu_1$,$(6,7)(7,6)$被分配到$\mu_2$

5. 收敛
    
\begin{figure}
    \centering
    \includegraphics*[scale=0.8]{img/a5_2.jpg}
    \caption{2.1}
\end{figure}

\subsection{}

1. $\mu_1=(1,2)\quad\mu_2=(3,4)$

2. $(1,1)(1,2)(2,1)$被分配到$\mu_1$,$(3,4)(6,7)(7,6)$被分配到$\mu_2$

3. $\mu_1=(1.33,1.33)\quad\mu_2=(5.33,5.67)$

4. $(1,1)(1,2)(2,1)$被分配到$\mu_1$,$(3,4)(6,7)(7,6)$被分配到$\mu_2$

5. 收敛

\begin{figure}
    \centering
    \includegraphics*[scale=0.8]{img/a5_3.jpg}
    \caption{2.2}
\end{figure}

\subsection{}

初始中心点的选择会影响算法的收敛过程,使得算法陷入局部最优解而不是全局最优解,导致聚类结果的不同。
图1的结果更优,因为$(3,4)$确实要离$(1,1)(1,2)(2,1)$这一簇更近一些


\section{问题三}

我们已知高斯混合模型中对软标签的更新过程为
\begin{align*}
    \gamma_{ik}=\frac{\pi_k N(x_i\vert\mu_k,\Sigma_k)}{\sum_{j=1}^K \pi_j N(x_i\vert\mu_k,\Sigma_k)}\\
\end{align*}
现在$\Sigma=\epsilon I$,展开正态分布
\begin{align*}
    \gamma_{ik}&=\frac{\pi_k \frac{1}{(2\pi^{n/2}\vert\Sigma\vert^{1/2})}
    \exp(-\frac{1}{2}(x_i-\mu_k)^T\Sigma^{-1}(x_i-\mu_k))}
    {\sum_{j=1}^K \pi_j \frac{1}{(2\pi^{n/2}\vert\Sigma\vert^{1/2})}
    \exp(-\frac{1}{2}(x_i-\mu_j)^T\Sigma^{-1}(x_i-\mu_j))}\\
    &=\frac{\pi_k\exp(-\frac{1}{2\epsilon}(x_i-\mu_k)^T(x_i-\mu_k))}
    {\sum_{j=1}^{K}\pi_j\exp(-\frac{1}{2\epsilon}(x_i-\mu_j)^T(x_i-\mu_j))}\\
    &=\frac{\pi_k\exp(-\frac{1}{2\epsilon}\Vert x_i-\mu_k\Vert^2)}
    {\sum_{j=1}^{K}\pi_j\exp(-\frac{1}{2\epsilon}\Vert x_i-\mu_j\Vert^2)}\\
\end{align*}
当$\epsilon\rightarrow0$时,对于数据$x_i$,假设其属于第$k$类的概率最大,那么该数据点与$\mu_k$的距离会非常近
即有$\Vert x_i-\mu_k\Vert^2\rightarrow0$,于是$\exp(-\frac{\Vert x_i-\mu_k\Vert^2}{2\epsilon})\rightarrow1$,
对任意$j\neq k$,$\exp(-\frac{\Vert x_i-\mu_k\Vert^2}{2\epsilon})\rightarrow0$,因此
\begin{align*}
    \gamma_{ik}=\left\{
        \begin{aligned}
            1 & \quad\text{if }k=\mathop{argmin}\limits_{j=1,\dots,K}\Vert x_i-\mu_j \Vert^2\\
            0 & \quad\text{otherwise}
        \end{aligned}
    \right.
\end{align*}
现在这与K-means的硬标签完全相同,可以证明当$\epsilon\rightarrow0$时,高斯混合模型与K-means等价

\section{问题四}

\subsection{}

\begin{align*}
    \gamma_k^{(i)}&=P(z_k^{(i)}=1\vert x^{(i)},\pi,\mathbf{p})\\
    &=\frac{P(x^{(i)}\vert z_k^{i}=1)P(z_k^{(i)}=1)}{\sum_{j=1}^KP(x^{(i)}\vert z_j^{i}=1)P(z_j^{(i)}=1)}\\
    &=\frac{\pi_k P(x^{(i)}\vert p^{(k)})}{\sum_{j=1}^K\pi_j P(x^{(i)}\vert p^{(j)})}
\end{align*}

\subsection{}

优化目标为
\begin{align*}
    \mathop{max}\limits_{\mathbf{p},\pi}\sum_{i=1}^{N}\sum_{k=1}^K \gamma_k^{(i)}(\log\pi_k+\log P(x^{(i)}\vert p^{(k)}))
    \quad \text{s.t.} \sum_{k=1}^K\pi_k=1
\end{align*}
构造拉格朗日方程
\begin{align*}
    L(\mathbf{p},\pi,\lambda)=\sum_{i=1}^{N}\sum_{k=1}^K \gamma_k^{(i)}(\log\pi_k+\log P(x^{(i)}\vert p^{(k)}))
    +\lambda(1-\sum_{k=1}^K\pi_k)
\end{align*}
计算$\pi_k$
\begin{align*}
    \frac{\partial L}{\partial \pi_k}&=\sum_{i=1}^N\frac{\gamma_k^{(i)}}{\pi_k}-\lambda=0\\
    \pi_k&=\frac{\sum_{i=1}^{N}\gamma_k^{(i)}}{\lambda}\\
\end{align*}
由于已知
\begin{align*}
    \lambda \sum_{k=1}^K\pi_k&=\sum_{i=1}^N\sum_{k=1}^K\gamma_{ik}=N\\
    &\Rightarrow \ \ \lambda=N
\end{align*}
因此
\begin{align*}
    \pi_k&=\frac{\sum_{i=1}^{N}\gamma_k^{(i)}}{N}\\
\end{align*}
计算$p^{(k)}$的第$d$个分量$p^{(k)}_d$
\begin{align*}
    \frac{\partial L}{\partial p^{(k)}_d}&=\sum_{i=1}^N \gamma_k^{(i)}\frac{\partial\log P(x^{(i)}_d\vert p^{(k)}_d)}{\partial p^{(k)}_d}\\
    &=\sum_{i=1}^N \gamma_k^{(i)}\frac{\partial\log\bigl((p^{(k)}_d)^{x^{(i)}_d}(1-p^{(k)}_d)^{1-x^{(i)}_d}\bigr)}{\partial p^{(k)}_d}\\
    &=\sum_{i=1}^N \gamma_k^{(i)}(\frac{x_d^{(i)}}{p^{(k)}_d}-\frac{1-x_d^{(i)}}{1-p_d^{(k)}})\\
    &=\sum_{i=1}^N \gamma_k^{(i)}(x_d^{(i)}-p_d^{(k)})=0\\
    p_d^{(k)}&=\frac{\sum_{i=1}^n\gamma_k^{(i)}x_d^{(i)}}{\sum_{i=1}^n\gamma_k^{(i)}}
\end{align*}
因此$p^{(k)}$为
\begin{align*}
    p^{(k)}&=\frac{\sum_{i=1}^n\gamma_k^{(i)}x^{(i)}}{\sum_{i=1}^n\gamma_k^{(i)}}
\end{align*}


\end{document}
