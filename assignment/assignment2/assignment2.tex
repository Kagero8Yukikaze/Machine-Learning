\documentclass[12pt, a4paper]{article}

\usepackage{ctex} % 使用ctex宏包支持中文
\usepackage{geometry} % 页面设置
\usepackage{amsmath, amssymb, ntheorem} % 数学公式
\usepackage{graphicx} % 插入图片
\usepackage{enumitem} % 列表环境
\usepackage{hyperref} % 超链接

% 页面设置
\geometry{left=2.5cm, right=2.5cm, top=2.5cm, bottom=2.5cm}

% 标题信息
\title{课后练习2}
\author{}
\date{}

\begin{document}

\maketitle % 显示标题

\section{问题一}

我们要求的是:
\begin{align*}
    &\mathop{min}\limits_{x,y,z}x^2+y^2+z^2\\
    \text{s.t.} \quad & x+y+z=9\\
    &x+2y+3z=20
\end{align*}

因此我们可以构造拉格朗日多项式:
\begin{equation*}
    L(x,y,z,\lambda_1,\lambda_2)=x^2+y^2+z^2 + \lambda_1(x+y+z-9) + \lambda_2(x+2y+3z-20)
\end{equation*}

对于使$f(x,y,z)$最小的$x^*,y^*,z^*$,我们有:
\begin{align*}
    &\nabla_x L(x^*,y^*,z^*,\lambda_1,\lambda_2)=0\\
    &\nabla_y L(x^*,y^*,z^*,\lambda_1,\lambda_2)=0\\
    &\nabla_z L(x^*,y^*,z^*,\lambda_1,\lambda_2)=0\\
    &\nabla_{\lambda_1} L(x^*,y^*,z^*,\lambda_1,\lambda_2)=0\\
    &\nabla_{\lambda_2} L(x^*,y^*,z^*,\lambda_1,\lambda_2)=0
\end{align*}

解得:
\begin{align*}
    &x^*=2\\
    &y^*=3\\
    &z^*=4\\
    &\lambda_1=\lambda_2=-2
\end{align*}

因此$\mathop{min}\limits_{x,y,z}f(x,y,z)=29$


\section{问题二}


\subsection{}

对于平面上的样本,它们都满足:
\begin{equation*}
    y_i(w^T x_i+b)\geq 1
\end{equation*}

取等号当且仅当$x_i$是支持向量. 现在已知$(6,5),(5,9),(2,4)$为支持向量,假设超平面为$w_1x_1+w_2x_2+b=0$,则有:
\begin{align*}
    &y_{(6,5)}(6w_1+5w_2+b)=1\\
    &y_{(5,9)}(5w_1+9w_2+b)=1\\
    &y_{(2,4)}(2w_1+4w_2+b)=1\\
\end{align*}

解得:
\begin{align*}
    &w=(\frac{8}{17},\frac{2}{17})^T\\
    &b=-\frac{41}{17}
\end{align*}

\subsection{}

对于$x_1=(1,15)$和$x_2=(4,2)$,有$w^Tx_1+b<0$和$w^Tx_2+b<0$. 因此两个新样本点都是负样本. 

\subsection{}

\newtheorem*{proof}{Proof}

\begin{proof}
    对于一个线性可分的数据分布,一定存在一个线性超平面$w^Tx+b=0$,使得正负样本能够完全位于超平面的两侧,
    其中支持向量是那些满足$y_i(w^Tx_i+b)=1$的向量.
    
    我们不妨假设正或负样本中有至少一者不存在支持向量,即不存在使得$y_i(w^Tx_i+b)=1$的向量.
    取正、负样本中$y_i(w^Tx_i+b)$值最小的向量$x_{j_1}$和$x_{j_2}$并记其值为$k_1$和$k_2$,且$k_1>k_2\geq 1$. 

    将该平面向靠近$x_{j_1}$方向平移$\frac{k_1-k_2}{2||w||}$的距离得到${w'}^Tx+b'=0$,则此时有
    \begin{align*}
        &y_{j_1}({w'}^Tx_{j_1}+b')\\
        =&y_{j_2}({w'}^Tx_{j_2}+b')\\
        =&\frac{k_1+k_2}{2}>1
    \end{align*}
    只需令
    \begin{align*}
        w''&=w'/(\frac{k_1+k_2}{2})\\
        b''&=b'/(\frac{k_1+k_2}{2})
    \end{align*}
    即可使$x_{j_1}$和$x_{j_2}$成为支持向量且此时有$||w''||^2<||w||^2$,
    这表明$w''$相比于$w$是一个更优解,说明假设不成立,则正负样本中均至少有一个支持向量. 
\end{proof}


\section{问题三}


\subsection{}

由于已知$\alpha_i > 0, \ \ \forall i$,故这6个样本点均为支持向量,因此可以计算得到:
\begin{align*}
    b&=\frac{1}{y_1}-\sum_{i=1}^{6}\alpha_i y_i \exp(-\frac{||x_i-x_1||^2}{2}) \\
    &\approx -0.07389
\end{align*}

\subsection{}

\begin{align*}
    f(x)=\sum_{i=1}^6 \alpha_i y_i \exp(-\frac{||x-x_i||^2}{2}) +b
\end{align*}

\subsection{}

对于样本$x_4=(2,2)$:
\begin{align*}
    f(x_4)=\sum_{i=1}^{6}\alpha_i y_i \exp(-\frac{||x_i-x_4||^2}{2})+b<0 
\end{align*}
故可以验证其是一个负样本

对于新的样本点$x_7=(-2,-2)$:
\begin{align*}
    f(x_7)=\sum_{i=1}^{6}\alpha_i y_i \exp(-\frac{||x_i-x_7||^2}{2})+b>0
\end{align*}
故其是一个正样本


\section{问题四}


\subsection{}

加入松弛变量前:
\begin{align*}
    &y_3(w^T x_3+b)=1\\
    &y_4(w^T x_4+b)=1
\end{align*}
因此$x_3$和$x_4$是支持向量

加入松弛变量后,由于$\xi_1=\xi_2=\xi_4=0$,而
\begin{align*}
    y_3({w'}^T x_3+b')\geq 1-\xi_3=-0.5
\end{align*}
因此这个问题转化为:
\begin{align*}
    &\min \frac{1}{2}||w||^2+C\sum_{i=1}^{4}\xi_i\\
    \text{s.t.} \quad &w^Tx_3+b=-0.5\\
    &w^Tx_4+b=-1\\
    &w^Tx_1+b \geq 1\\
    &w^Tx_2+b \geq 1
\end{align*}
解得:
\begin{align*}
    &w=(\frac{1}{2},\frac{1}{2})^T\\
    &b=-2
\end{align*}
故新的分隔超平面为:
\begin{align*}
    0.5x^{(1)}+0.5x^{(2)}-2=0
\end{align*}
此时$x_2$, $x_3$和$x_4$为支持向量

\subsection{}

硬间隔为:
\begin{align*}
    \frac{|w^Tx_3+b|}{||w||}=\frac{\sqrt{2}}{4}
\end{align*}

软间隔为:
\begin{align*}
    \frac{|{w'}^T x_2+b'|}{||w'||}=\sqrt{2}
\end{align*}

间隔变大,模型的鲁棒性更高,对噪声的抗干扰能力更强,对新加入的样本也不容易误判


\subsection{}

设四个数据点的Hinge Loss为$L_1, \dots, L_4$,则:
\begin{align*}
    &L_1=\max\{0,1-y_1({w'}^T x_1+b')\}=0\\
    &L_2=\max\{0,1-y_2({w'}^T x_2+b')\}=0\\
    &L_3=\max\{0,1-y_3({w'}^T x_3+b')\}=1.5\\
    &L_4=\max\{0,1-y_4({w'}^T x_4+b')\}=0\\
\end{align*}

只有分错的点和非常靠近超平面的点的Hinge Loss才不会为0,其余数据点均为0,
不用考虑,即只有Loss不为0的点才会对模型产生影响,而这些点相对于整个样本空间是很少的,
因此模型的解的稀疏性得到了保证

\end{document}
