\documentclass[12pt, a4paper]{article}

\usepackage{ctex} % 使用ctex宏包支持中文
\usepackage{geometry} % 页面设置
\usepackage{amsmath, amssymb, ntheorem} % 数学公式
\usepackage{graphicx} % 插入图片
\usepackage{enumitem} % 列表环境
\usepackage{hyperref} % 超链接

% 页面设置
\geometry{left=2.5cm, right=2.5cm, top=2.5cm, bottom=2.5cm}

% 标题信息
\title{课后练习1}
\author{}
\date{}

\begin{document}

\maketitle % 显示标题

\section{问题一}

我们要求的是:
\begin{align*}
    &\mathop{min}\limits_{x,y,z}x^2+y^2+z^2\\
    \text{s.t.} \quad & x+y+z=9\\
    &x+2y+3z=20
\end{align*}

因此我们可以构造拉格朗日多项式:
\begin{equation*}
    L(x,y,z,\lambda_1,\lambda_2)=x^2+y^2+z^2 + \lambda_1(x+y+z-9) + \lambda_2(x+2y+3z-20)
\end{equation*}

对于使$f(x,y,z)$最小的$x^*,y^*,z^*$,我们有:
\begin{align*}
    &\nabla_x L(x^*,y^*,z^*,\lambda_1,\lambda_2)=0\\
    &\nabla_y L(x^*,y^*,z^*,\lambda_1,\lambda_2)=0\\
    &\nabla_z L(x^*,y^*,z^*,\lambda_1,\lambda_2)=0\\
    &\nabla_{\lambda_1} L(x^*,y^*,z^*,\lambda_1,\lambda_2)=0\\
    &\nabla_{\lambda_2} L(x^*,y^*,z^*,\lambda_1,\lambda_2)=0
\end{align*}

解得:
\begin{align*}
    &x^*=3\\
    &y^*=4\\
    &z^*=5\\
    &\lambda_1=\lambda_2=-2
\end{align*}

因此$\mathop{min}\limits_{x,y,z}f(x,y,z)=50$


\section{问题二}


\subsection{}

对于平面上的样本,它们都满足:
\begin{equation*}
    y_i(w^T x_i+b)\geq 1
\end{equation*}

取等号当且仅当$x_i$是支持向量. 现在已知$(6,5),(5,9),(2,4)$为支持向量,假设超平面为$w_1x_1+w_2x_2+b=0$,则有:
\begin{align*}
    &y_{(6,5)}(6w_1+5w_2+b)=1\\
    &y_{(5,9)}(5w_1+9w_2+b)=1\\
    &y_{(2,4)}(2w_1+4w_2+b)=1\\
\end{align*}

解得:
\begin{align*}
    &w=(\frac{8}{17},\frac{2}{17})^T\\
    &b=-\frac{41}{17}
\end{align*}

\subsection{}

对于$x_1=(1,15)$和$x_2=(4,2)$,有$wx_1+b<0$和$wx_2+b<0$. 因此两个新样本点都是负样本. 

\subsection{}



\section{问题三}


\subsection{}

由于已知$\alpha_i > 0, \ \ \forall i$,故这6个样本点均为支持向量,首先计算$w$:
\begin{align*}
    w&=\sum_{i=1}^{6}\alpha_i y_i x_i\\
    &=(0, -2.1)^T
\end{align*}

因此可以计算得到:
\begin{align*}
    b&=\frac{1}{y_1}-w^T x_1\\
    &=3.01
\end{align*}

\subsection{}


\subsection{}



\section{问题四}


\subsection{}



\subsection{}


\subsection{}



\end{document}
