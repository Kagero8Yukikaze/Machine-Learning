\documentclass[12pt, a4paper]{article}

\usepackage{ctex} % 使用ctex宏包支持中文
\usepackage{geometry} % 页面设置
\usepackage{amsmath, amssymb} % 数学公式
\usepackage{graphicx} % 插入图片
\usepackage{enumitem} % 列表环境
\usepackage{hyperref} % 超链接

% 页面设置
\geometry{left=2.5cm, right=2.5cm, top=2.5cm, bottom=2.5cm}

% 标题信息
\title{课后练习1}
\author{-}

\begin{document}

\maketitle % 显示标题

\section{问题一}

\subsection{}

令$x = \begin{bmatrix}
    0 & 1 & 1 \\
    2 & 2 & 1 \\
    3 & 4 & 1
\end{bmatrix}$, $y = \begin{bmatrix}
    1 \\ 3 \\ 4
\end{bmatrix}$, $\hat{w} = \begin{bmatrix}
    w_1 \\ w_2 \\ b
\end{bmatrix}$, 可以计算$x^T x = \begin{bmatrix}
    13 & 16 & 5\\
    16 & 21 & 7 \\
    5 & 7 & 3
\end{bmatrix}$


那么$\hat{w} = (x^T x)^{-1}x^T y = \begin{bmatrix}
    1 \\
    0 \\
    1
\end{bmatrix}$, 故$w_1 = 1, \ w_2 = 0, \ b = 1$

\subsection{}

此时$x = \begin{bmatrix}
    0&1&2&1\\
    2&2&4&1\\
    3&4&8&1
\end{bmatrix}$, $\hat{w} = \begin{bmatrix}
    w_1\\w_2\\w_3\\b
\end{bmatrix}$, 计算$x^Tx = \begin{bmatrix}
    13&16&32&5\\
    16&21&42&7\\
    32&42&84&14\\
    5&7&14&3
\end{bmatrix}$, 可以发现这是一个奇异矩阵(有重复的行), 没有逆矩阵, 因此此时线性回归没有唯一解

\subsection{}

已知$\lambda = 1$, 故$x^Tx + \lambda I = \begin{bmatrix}
    14&16&32&5\\
    16&22&42&7\\
    32&42&85&14\\
    5&7&14&4
\end{bmatrix}$

那么$\hat{w} = (x^T x + \lambda I)^{-1}x^T y = \begin{bmatrix}
    0.378\\
    0.140\\
    0.280\\
    0.304
\end{bmatrix}$, 因此可得$w_1=0.378, \ w_2=0.140, \ w_3 = 0.280, \ b = 0.304$

\section{问题二}

这里是问题二的描述。

\subsection{解答}

这里是问题二的解答。

\section{问题三}

这里是问题三的描述。

\subsection{解答}

这里是问题三的解答。

\section{问题四}

\end{document}
